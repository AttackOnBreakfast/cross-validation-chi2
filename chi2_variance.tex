\documentclass{article}
\usepackage{amsmath}
\usepackage{amsfonts}

\begin{document}

\section*{Variance of Cross-Validated $\chi^2$}

To understand the uncertainty associated with the empirical $\chi^2$ values for both the training and test datasets, we compute the \textbf{empirical variance} across repeated trials, and compare it to the \textbf{theoretical variance} predicted by chi-squared statistics.

\subsection*{Empirical Variance Calculation}

For each model complexity (polynomial degree) $m$, and each of the $T$ random trials, we compute:

\[
\chi^2_A^{(t)} = \frac{1}{\sigma^2} \sum_{i=1}^N (y_{A,i}^{(t)} - \hat{y}_{A,i}^{(t)})^2, \quad t = 1, 2, \dots, T
\]

Then the sample mean and variance are:

\[
\overline{\chi^2_A}(m) = \frac{1}{T} \sum_{t=1}^T \chi^2_A^{(t)}(m)
\]
\[
\mathrm{Var}(\chi^2_A)(m) = \frac{1}{T} \sum_{t=1}^T \left( \chi^2_A^{(t)}(m) - \overline{\chi^2_A}(m) \right)^2
\]

The same formulas hold for the test dataset $\chi^2_B$.

\subsection*{Theoretical Variance of Chi-Squared Distribution}

If the model is correct and the residuals are Gaussian, then the quantity $\chi^2$ follows a chi-squared distribution with $\nu$ degrees of freedom. The mean and variance of a chi-squared random variable with $\nu$ degrees of freedom are:

\[
\mathbb{E}[\chi^2] = \nu, \quad \mathrm{Var}[\chi^2] = 2\nu
\]

We apply this to estimate the theoretical variance for both datasets:

\begin{align*}
\mathrm{Var}_{\text{theory}}(\chi^2_A) &= 2(N - m) \\
\mathrm{Var}_{\text{theory}}(\chi^2_B) &= 2(N + m)
\end{align*}

Here, $N$ is the number of data points in each dataset, and $m$ is the number of fit parameters (i.e., polynomial degree + 1).

\subsection*{Comparison and Visualization}

In the plot:
\begin{itemize}
    \item The empirical standard deviations $\sqrt{\mathrm{Var}(\chi^2_A)}$ and $\sqrt{\mathrm{Var}(\chi^2_B)}$ are shown as vertical error bars.
    \item The theoretical standard deviations $\sqrt{2(N - m)}$ and $\sqrt{2(N + m)}$ are shown as shaded bands centered on the average $\chi^2$ values.
\end{itemize}

This comparison helps evaluate whether the observed variability in $\chi^2$ aligns with expectations under the assumption of Gaussian noise and a well-calibrated model.

\end{document}